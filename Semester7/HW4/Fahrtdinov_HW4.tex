\documentclass{article}\usepackage[]{graphicx}\usepackage[]{color}
% maxwidth is the original width if it is less than linewidth
% otherwise use linewidth (to make sure the graphics do not exceed the margin)
\makeatletter
\def\maxwidth{ %
  \ifdim\Gin@nat@width>\linewidth
    \linewidth
  \else
    \Gin@nat@width
  \fi
}
\makeatother

\definecolor{fgcolor}{rgb}{0.345, 0.345, 0.345}
\newcommand{\hlnum}[1]{\textcolor[rgb]{0.686,0.059,0.569}{#1}}%
\newcommand{\hlstr}[1]{\textcolor[rgb]{0.192,0.494,0.8}{#1}}%
\newcommand{\hlcom}[1]{\textcolor[rgb]{0.678,0.584,0.686}{\textit{#1}}}%
\newcommand{\hlopt}[1]{\textcolor[rgb]{0,0,0}{#1}}%
\newcommand{\hlstd}[1]{\textcolor[rgb]{0.345,0.345,0.345}{#1}}%
\newcommand{\hlkwa}[1]{\textcolor[rgb]{0.161,0.373,0.58}{\textbf{#1}}}%
\newcommand{\hlkwb}[1]{\textcolor[rgb]{0.69,0.353,0.396}{#1}}%
\newcommand{\hlkwc}[1]{\textcolor[rgb]{0.333,0.667,0.333}{#1}}%
\newcommand{\hlkwd}[1]{\textcolor[rgb]{0.737,0.353,0.396}{\textbf{#1}}}%
\let\hlipl\hlkwb

\usepackage{framed}
\makeatletter
\newenvironment{kframe}{%
 \def\at@end@of@kframe{}%
 \ifinner\ifhmode%
  \def\at@end@of@kframe{\end{minipage}}%
  \begin{minipage}{\columnwidth}%
 \fi\fi%
 \def\FrameCommand##1{\hskip\@totalleftmargin \hskip-\fboxsep
 \colorbox{shadecolor}{##1}\hskip-\fboxsep
     % There is no \\@totalrightmargin, so:
     \hskip-\linewidth \hskip-\@totalleftmargin \hskip\columnwidth}%
 \MakeFramed {\advance\hsize-\width
   \@totalleftmargin\z@ \linewidth\hsize
   \@setminipage}}%
 {\par\unskip\endMakeFramed%
 \at@end@of@kframe}
\makeatother

\definecolor{shadecolor}{rgb}{.97, .97, .97}
\definecolor{messagecolor}{rgb}{0, 0, 0}
\definecolor{warningcolor}{rgb}{1, 0, 1}
\definecolor{errorcolor}{rgb}{1, 0, 0}
\newenvironment{knitrout}{}{} % an empty environment to be redefined in TeX

\usepackage{alltt}
\PassOptionsToPackage{unicode}{hyperref}
\PassOptionsToPackage{naturalnames}{hyperref}
\usepackage{fullpage}
\usepackage[T2A]{fontenc}
\usepackage[utf8]{inputenc}
\usepackage[russian]{babel}
\usepackage{mathrsfs}
\usepackage{amsfonts}
\usepackage{amsmath }
\IfFileExists{upquote.sty}{\usepackage{upquote}}{}
\begin{document}
\title{Отчет по домашнему заданию}
\pretitle{\vspace{\droptitle}\centering\huge}
\posttitle{\par}
\author{Фахртдинов Т. А.}


\maketitle
Четвертая задача. Статистический анализ категориальных(зависимых) признаков.

3 вариант.
\begin{knitrout}
\definecolor{shadecolor}{rgb}{0.969, 0.969, 0.969}\color{fgcolor}\begin{kframe}
\begin{alltt}
\hlstd{SH1} \hlkwb{<-} \hlkwd{c}\hlstd{(}\hlnum{8.1}\hlstd{,} \hlnum{6.2}\hlstd{,} \hlnum{6.11}\hlstd{,} \hlnum{5.23}\hlstd{,} \hlnum{5.2}\hlstd{,} \hlnum{8.12}\hlstd{,} \hlnum{6.21}\hlstd{,} \hlnum{6.21}\hlstd{,} \hlnum{5}\hlstd{,} \hlnum{6}\hlstd{,} \hlnum{8.16}\hlstd{,} \hlnum{4.96}\hlstd{,} \hlnum{5.36}\hlstd{,} \hlnum{5.92}\hlstd{,} \hlnum{8.32}
         \hlstd{,} \hlnum{8.08}\hlstd{,} \hlnum{6.52}\hlstd{,} \hlnum{5.2}\hlstd{,} \hlnum{7}\hlstd{,} \hlnum{5}\hlstd{,} \hlnum{6.3}\hlstd{,} \hlnum{6.1}\hlstd{,} \hlnum{5.95}\hlstd{,} \hlnum{5.2}\hlstd{,} \hlnum{8.5}\hlstd{,} \hlnum{8.2}\hlstd{,} \hlnum{7.3}\hlstd{,} \hlnum{6.45}\hlstd{,} \hlnum{5.2}\hlstd{,} \hlnum{6.2}
         \hlstd{,} \hlnum{4.95}\hlstd{,} \hlnum{6.45}\hlstd{,} \hlnum{4.95}\hlstd{,} \hlnum{6.55}\hlstd{,} \hlnum{6.5}\hlstd{,} \hlnum{5.9}\hlstd{,} \hlnum{6.2}\hlstd{,} \hlnum{6.5}\hlstd{,} \hlnum{7.7}\hlstd{,} \hlnum{7.2}\hlstd{,} \hlnum{7}\hlstd{,} \hlnum{7.4}\hlstd{,} \hlnum{7.8}\hlstd{,} \hlnum{7.2}\hlstd{,} \hlnum{7}\hlstd{)}

\hlstd{SH2} \hlkwb{<-} \hlkwd{c}\hlstd{(}\hlnum{8.9}\hlstd{,} \hlnum{8.16}\hlstd{,} \hlnum{8.1}\hlstd{,} \hlnum{7.44}\hlstd{,} \hlnum{7.46}\hlstd{,} \hlnum{8.91}\hlstd{,} \hlnum{8.1}\hlstd{,}\hlnum{8.41}\hlstd{,} \hlnum{7.44}\hlstd{,} \hlnum{8.43}\hlstd{,} \hlnum{8.92}\hlstd{,} \hlnum{6.48}\hlstd{,} \hlnum{7.2}\hlstd{,} \hlnum{7.4}\hlstd{,} \hlnum{8.9}
         \hlstd{,} \hlnum{8.9}\hlstd{,} \hlnum{8.1}\hlstd{,} \hlnum{6.88}\hlstd{,} \hlnum{8.6}\hlstd{,} \hlnum{7.2}\hlstd{,} \hlnum{8.4}\hlstd{,} \hlnum{8.4}\hlstd{,} \hlnum{8.4}\hlstd{,} \hlnum{7.4}\hlstd{,} \hlnum{8.91}\hlstd{,} \hlnum{8.9}\hlstd{,} \hlnum{8.8}\hlstd{,} \hlnum{8.1}\hlstd{,} \hlnum{7.4}\hlstd{,} \hlnum{8.4}
         \hlstd{,} \hlnum{7.3}\hlstd{,} \hlnum{7.9}\hlstd{,} \hlnum{7.36}\hlstd{,} \hlnum{8.1}\hlstd{,} \hlnum{8}\hlstd{,} \hlnum{8.4} \hlstd{,} \hlnum{8.2}\hlstd{,} \hlnum{8.3}\hlstd{,} \hlnum{8}\hlstd{,} \hlnum{8.1}\hlstd{,} \hlnum{8.2}\hlstd{,} \hlnum{8.6}\hlstd{,} \hlnum{8.5}\hlstd{,} \hlnum{8.6}\hlstd{,} \hlnum{8.5}\hlstd{)}
\end{alltt}
\end{kframe}
\end{knitrout}

Строим таблицу:
\begin{knitrout}
\definecolor{shadecolor}{rgb}{0.969, 0.969, 0.969}\color{fgcolor}\begin{kframe}
\begin{alltt}
\hlstd{tab} \hlkwb{<-} \hlkwd{matrix}\hlstd{(}\hlkwd{c}\hlstd{(}\hlnum{0}\hlstd{,} \hlnum{0}\hlstd{,} \hlnum{0}\hlstd{,} \hlnum{0}\hlstd{),} \hlkwc{nrow} \hlstd{=} \hlnum{2}\hlstd{)}
\hlstd{tab[}\hlnum{1}\hlstd{,} \hlnum{1}\hlstd{]} \hlkwb{<-} \hlkwd{length}\hlstd{(SH1[SH1} \hlopt{<=} \hlkwd{median}\hlstd{(SH1)])}
\hlstd{tab[}\hlnum{2}\hlstd{,} \hlnum{1}\hlstd{]} \hlkwb{<-} \hlkwd{length}\hlstd{(SH1[SH1} \hlopt{>} \hlkwd{median}\hlstd{(SH1)])}
\hlstd{tab[}\hlnum{1}\hlstd{,} \hlnum{2}\hlstd{]} \hlkwb{<-} \hlkwd{length}\hlstd{(SH2[SH2} \hlopt{<=} \hlkwd{median}\hlstd{(SH1)])}
\hlstd{tab[}\hlnum{2}\hlstd{,} \hlnum{2}\hlstd{]} \hlkwb{<-} \hlkwd{length}\hlstd{(SH2[SH2} \hlopt{>} \hlkwd{median}\hlstd{(SH1)])}
\end{alltt}
\end{kframe}
\end{knitrout}
Получаем:
\begin{knitrout}
\definecolor{shadecolor}{rgb}{0.969, 0.969, 0.969}\color{fgcolor}\begin{kframe}
\begin{verbatim}
##           Sum
##     23  0  23
##     22 45  67
## Sum 45 45  90
\end{verbatim}
\end{kframe}
\end{knitrout}
Нас интересует насколько значимо различие между частотами значения в ячейке $[1, 2]$ и $[2, 1]$.

$H_0:$ tab[1, 2] = tab[2, 1] (Ситуация улучшения и ухудшения после лечения равновероятны)

Посчитаем точную статистику критерия Мак Немара:
\begin{knitrout}
\definecolor{shadecolor}{rgb}{0.969, 0.969, 0.969}\color{fgcolor}\begin{kframe}
\begin{alltt}
\hlstd{a} \hlkwb{<-} \hlstd{tab[}\hlnum{1}\hlstd{,} \hlnum{2}\hlstd{]}
\hlstd{b} \hlkwb{<-} \hlstd{tab[}\hlnum{2}\hlstd{,} \hlnum{1}\hlstd{]}
\hlstd{alfa} \hlkwb{<-} \hlnum{0}
\hlkwa{for} \hlstd{(i} \hlkwa{in} \hlnum{0}\hlopt{:}\hlkwd{min}\hlstd{(a, b))}
\hlstd{\{}
  \hlstd{alfa} \hlkwb{<-} \hlstd{alfa} \hlopt{+} \hlkwd{choose}\hlstd{(a} \hlopt{+} \hlstd{b, i)} \hlopt{*} \hlstd{(}\hlnum{1} \hlopt{/} \hlnum{2}\hlopt{^}\hlstd{(a} \hlopt{+} \hlstd{b))}
\hlstd{\}}
\hlstd{alfa} \hlkwb{<-} \hlnum{2} \hlopt{*} \hlstd{alfa}
\hlstd{alfa}
\end{alltt}
\begin{verbatim}
## [1] 4.768372e-07
\end{verbatim}
\end{kframe}
\end{knitrout}
alfa < 0.05, отклоняем гипотезу $H_0$, о том, что ситуация улучшения и ухудшения после лечения равновероятны.

Проверим результат используя функцию из пакета exact2x2 mcnemar.exact:
\begin{knitrout}
\definecolor{shadecolor}{rgb}{0.969, 0.969, 0.969}\color{fgcolor}\begin{kframe}


{\ttfamily\noindent\itshape\color{messagecolor}{\#\# Loading required package: exactci}}

{\ttfamily\noindent\itshape\color{messagecolor}{\#\# Loading required package: ssanv}}\end{kframe}
\end{knitrout}
\begin{knitrout}
\definecolor{shadecolor}{rgb}{0.969, 0.969, 0.969}\color{fgcolor}\begin{kframe}
\begin{alltt}
\hlkwd{mcnemar.exact}\hlstd{(tab)}
\end{alltt}
\begin{verbatim}
## 
## 	Exact McNemar test (with central confidence intervals)
## 
## data:  tab
## b = 0, c = 22, p-value = 4.768e-07
## alternative hypothesis: true odds ratio is not equal to 1
## 95 percent confidence interval:
##  0.0000000 0.1825538
## sample estimates:
## odds ratio 
##          0
\end{verbatim}
\end{kframe}
\end{knitrout}
Значение критерия совпадает с посчитанным нами.




\end{document}
