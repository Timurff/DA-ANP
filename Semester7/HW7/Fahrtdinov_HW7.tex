\documentclass{article}\usepackage[]{graphicx}\usepackage[]{color}
% maxwidth is the original width if it is less than linewidth
% otherwise use linewidth (to make sure the graphics do not exceed the margin)
\makeatletter
\def\maxwidth{ %
  \ifdim\Gin@nat@width>\linewidth
    \linewidth
  \else
    \Gin@nat@width
  \fi
}
\makeatother

\definecolor{fgcolor}{rgb}{0.345, 0.345, 0.345}
\newcommand{\hlnum}[1]{\textcolor[rgb]{0.686,0.059,0.569}{#1}}%
\newcommand{\hlstr}[1]{\textcolor[rgb]{0.192,0.494,0.8}{#1}}%
\newcommand{\hlcom}[1]{\textcolor[rgb]{0.678,0.584,0.686}{\textit{#1}}}%
\newcommand{\hlopt}[1]{\textcolor[rgb]{0,0,0}{#1}}%
\newcommand{\hlstd}[1]{\textcolor[rgb]{0.345,0.345,0.345}{#1}}%
\newcommand{\hlkwa}[1]{\textcolor[rgb]{0.161,0.373,0.58}{\textbf{#1}}}%
\newcommand{\hlkwb}[1]{\textcolor[rgb]{0.69,0.353,0.396}{#1}}%
\newcommand{\hlkwc}[1]{\textcolor[rgb]{0.333,0.667,0.333}{#1}}%
\newcommand{\hlkwd}[1]{\textcolor[rgb]{0.737,0.353,0.396}{\textbf{#1}}}%
\let\hlipl\hlkwb

\usepackage{framed}
\makeatletter
\newenvironment{kframe}{%
 \def\at@end@of@kframe{}%
 \ifinner\ifhmode%
  \def\at@end@of@kframe{\end{minipage}}%
  \begin{minipage}{\columnwidth}%
 \fi\fi%
 \def\FrameCommand##1{\hskip\@totalleftmargin \hskip-\fboxsep
 \colorbox{shadecolor}{##1}\hskip-\fboxsep
     % There is no \\@totalrightmargin, so:
     \hskip-\linewidth \hskip-\@totalleftmargin \hskip\columnwidth}%
 \MakeFramed {\advance\hsize-\width
   \@totalleftmargin\z@ \linewidth\hsize
   \@setminipage}}%
 {\par\unskip\endMakeFramed%
 \at@end@of@kframe}
\makeatother

\definecolor{shadecolor}{rgb}{.97, .97, .97}
\definecolor{messagecolor}{rgb}{0, 0, 0}
\definecolor{warningcolor}{rgb}{1, 0, 1}
\definecolor{errorcolor}{rgb}{1, 0, 0}
\newenvironment{knitrout}{}{} % an empty environment to be redefined in TeX

\usepackage{alltt}
\PassOptionsToPackage{unicode}{hyperref}
\PassOptionsToPackage{naturalnames}{hyperref}
\usepackage{fullpage}
\usepackage[T2A]{fontenc}
\usepackage[utf8]{inputenc}
\usepackage[russian]{babel}
\usepackage{mathrsfs}
\usepackage{amsfonts}
\usepackage{amsmath }
\IfFileExists{upquote.sty}{\usepackage{upquote}}{}
\begin{document}
\title{Отчет по домашнему заданию}
\pretitle{\vspace{\droptitle}\centering\huge}
\posttitle{\par}
\author{Фахртдинов Т. А.}


\maketitle
Седьмая задача. Проверка гипотез однородности для зависимых
выборок.

Вариант 3.

\textbf{При помощи параметрического критерия Стьюдента для зависимых выборок выясним, значимо ли изменение в клинических показателях больных при поступлении в стационар и при их выписке.}
\begin{knitrout}
\definecolor{shadecolor}{rgb}{0.969, 0.969, 0.969}\color{fgcolor}\begin{kframe}
\begin{alltt}
\hlstd{SH1} \hlkwb{<-} \hlkwd{c}\hlstd{(}\hlnum{8.1}\hlstd{,} \hlnum{6.2}\hlstd{,} \hlnum{6.11}\hlstd{,} \hlnum{5.23}\hlstd{,} \hlnum{5.2}\hlstd{,} \hlnum{8.12}\hlstd{,} \hlnum{6.21}\hlstd{,} \hlnum{6.21}\hlstd{,} \hlnum{5}\hlstd{,} \hlnum{6}\hlstd{,} \hlnum{8.16}\hlstd{,} \hlnum{4.96}\hlstd{,} \hlnum{5.36}\hlstd{,} \hlnum{5.92}\hlstd{,} \hlnum{8.32}
         \hlstd{,} \hlnum{8.08}\hlstd{,} \hlnum{6.52}\hlstd{,} \hlnum{5.2}\hlstd{,} \hlnum{7}\hlstd{,} \hlnum{5}\hlstd{,} \hlnum{6.3}\hlstd{,} \hlnum{6.1}\hlstd{,} \hlnum{5.95}\hlstd{,} \hlnum{5.2}\hlstd{,} \hlnum{8.5}\hlstd{,} \hlnum{8.2}\hlstd{,} \hlnum{7.3}\hlstd{,} \hlnum{6.45}\hlstd{,} \hlnum{5.2}\hlstd{,} \hlnum{6.2}
         \hlstd{,} \hlnum{4.95}\hlstd{,} \hlnum{6.45}\hlstd{,} \hlnum{4.95}\hlstd{,} \hlnum{6.55}\hlstd{,} \hlnum{6.5}\hlstd{,} \hlnum{5.9}\hlstd{,} \hlnum{6.2}\hlstd{,} \hlnum{6.5}\hlstd{,} \hlnum{7.7}\hlstd{,} \hlnum{7.2}\hlstd{,} \hlnum{7}\hlstd{,} \hlnum{7.4}\hlstd{,} \hlnum{7.8}\hlstd{,} \hlnum{7.2}\hlstd{,} \hlnum{7}\hlstd{)}

\hlstd{SH2} \hlkwb{<-} \hlkwd{c}\hlstd{(}\hlnum{8.9}\hlstd{,} \hlnum{8.16}\hlstd{,} \hlnum{8.1}\hlstd{,} \hlnum{7.44}\hlstd{,} \hlnum{7.46}\hlstd{,} \hlnum{8.91}\hlstd{,} \hlnum{8.1}\hlstd{,}\hlnum{8.41}\hlstd{,} \hlnum{7.44}\hlstd{,} \hlnum{8.43}\hlstd{,} \hlnum{8.92}\hlstd{,} \hlnum{6.48}\hlstd{,} \hlnum{7.2}\hlstd{,} \hlnum{7.4}\hlstd{,} \hlnum{8.9}
         \hlstd{,} \hlnum{8.9}\hlstd{,} \hlnum{8.1}\hlstd{,} \hlnum{6.88}\hlstd{,} \hlnum{8.6}\hlstd{,} \hlnum{7.2}\hlstd{,} \hlnum{8.4}\hlstd{,} \hlnum{8.4}\hlstd{,} \hlnum{8.4}\hlstd{,} \hlnum{7.4}\hlstd{,} \hlnum{8.91}\hlstd{,} \hlnum{8.9}\hlstd{,} \hlnum{8.8}\hlstd{,} \hlnum{8.1}\hlstd{,} \hlnum{7.4}\hlstd{,} \hlnum{8.4}
         \hlstd{,} \hlnum{7.3}\hlstd{,} \hlnum{7.9}\hlstd{,} \hlnum{7.36}\hlstd{,} \hlnum{8.1}\hlstd{,} \hlnum{8}\hlstd{,} \hlnum{8.4} \hlstd{,} \hlnum{8.2}\hlstd{,} \hlnum{8.3}\hlstd{,} \hlnum{8}\hlstd{,} \hlnum{8.1}\hlstd{,} \hlnum{8.2}\hlstd{,} \hlnum{8.6}\hlstd{,} \hlnum{8.5}\hlstd{,} \hlnum{8.6}\hlstd{,} \hlnum{8.5}\hlstd{)}

\hlstd{SH} \hlkwb{<-} \hlstd{SH1} \hlopt{-} \hlstd{SH2}
\hlstd{T} \hlkwb{<-} \hlkwd{mean}\hlstd{(SH)} \hlopt{*} \hlkwd{sqrt}\hlstd{(}\hlkwd{length}\hlstd{(SH))} \hlopt{/} \hlkwd{sqrt}\hlstd{(}\hlkwd{var}\hlstd{(SH))}
\end{alltt}
\end{kframe}
\end{knitrout}
Значение критерия и p-value:
\begin{knitrout}
\definecolor{shadecolor}{rgb}{0.969, 0.969, 0.969}\color{fgcolor}\begin{kframe}
\begin{verbatim}
## [1] -1.743422e+01  2.229870e-21
\end{verbatim}
\end{kframe}
\end{knitrout}
Найдем значение критерия с помощью встроенной функции:
\begin{knitrout}
\definecolor{shadecolor}{rgb}{0.969, 0.969, 0.969}\color{fgcolor}\begin{kframe}
\begin{alltt}
\hlkwd{t.test}\hlstd{(SH)}
\end{alltt}
\begin{verbatim}
## 
## 	One Sample t-test
## 
## data:  SH
## t = -17.434, df = 44, p-value < 2.2e-16
## alternative hypothesis: true mean is not equal to 0
## 95 percent confidence interval:
##  -1.822144 -1.444523
## sample estimates:
## mean of x 
## -1.633333
\end{verbatim}
\end{kframe}
\end{knitrout}
Значение, которое получили мы совпало со значением встроенной функции.

p-value < 0.05, отклоняем гипотезу о равенстве средних.

Для SH1, среднее \textpm ошибка среднего: 6.48 \textpm 0.159.

Для SH2, среднее \textpm ошибка среднего: 8.113 \textpm 0.091.

\newpage

\textbf{Уменьшается или увеличивается преступность. Cochren Q тест:}

$H_0:$ Уровень преступности не изменился.
\begin{knitrout}
\definecolor{shadecolor}{rgb}{0.969, 0.969, 0.969}\color{fgcolor}\begin{kframe}
\begin{alltt}
\hlstd{W1} \hlkwb{<-} \hlkwd{c}\hlstd{(}\hlnum{60}\hlstd{,} \hlnum{8}\hlstd{,} \hlnum{34}\hlstd{,} \hlnum{31}\hlstd{,} \hlnum{21}\hlstd{,} \hlnum{42}\hlstd{,} \hlnum{2}\hlstd{,} \hlnum{6}\hlstd{,} \hlnum{21}\hlstd{,} \hlnum{2}\hlstd{)}
\hlstd{W2} \hlkwb{<-} \hlkwd{c}\hlstd{(}\hlnum{61}\hlstd{,} \hlnum{7}\hlstd{,} \hlnum{46}\hlstd{,} \hlnum{24}\hlstd{,} \hlnum{21}\hlstd{,} \hlnum{45}\hlstd{,} \hlnum{2}\hlstd{,} \hlnum{1}\hlstd{,} \hlnum{23}\hlstd{,} \hlnum{0}\hlstd{)}
\hlstd{W3} \hlkwb{<-} \hlkwd{c}\hlstd{(}\hlnum{57}\hlstd{,} \hlnum{12}\hlstd{,} \hlnum{44}\hlstd{,} \hlnum{12}\hlstd{,} \hlnum{11}\hlstd{,} \hlnum{46}\hlstd{,} \hlnum{2}\hlstd{,} \hlnum{4}\hlstd{,} \hlnum{17}\hlstd{,} \hlnum{2}\hlstd{)}

\hlstd{m} \hlkwb{<-} \hlkwd{median}\hlstd{(W1)}

\hlstd{W1} \hlkwb{<-} \hlkwd{as.numeric}\hlstd{(W1} \hlopt{>} \hlstd{m)}
\hlstd{W2} \hlkwb{<-} \hlkwd{as.numeric}\hlstd{(W2} \hlopt{>} \hlstd{m)}
\hlstd{W3} \hlkwb{<-} \hlkwd{as.numeric}\hlstd{(W3} \hlopt{>} \hlstd{m)}

\hlstd{s} \hlkwb{<-} \hlnum{3}
\hlstd{N} \hlkwb{<-} \hlkwd{sum}\hlstd{(W1)} \hlopt{+} \hlkwd{sum}\hlstd{(W2)} \hlopt{+} \hlkwd{sum}\hlstd{(W3)}

\hlstd{a} \hlkwb{<-} \hlstd{(}\hlkwd{sum}\hlstd{(W1)} \hlopt{-} \hlstd{N} \hlopt{/} \hlstd{s)}\hlopt{^}\hlnum{2} \hlopt{+} \hlstd{(}\hlkwd{sum}\hlstd{(W2)} \hlopt{-} \hlstd{N} \hlopt{/} \hlstd{s)}\hlopt{^}\hlnum{2} \hlopt{+} \hlstd{(}\hlkwd{sum}\hlstd{(W3)} \hlopt{-} \hlstd{N} \hlopt{/} \hlstd{s)}\hlopt{^}\hlnum{2}
\hlstd{b} \hlkwb{<-} \hlnum{0}
\hlkwa{for}\hlstd{(i} \hlkwa{in} \hlnum{1}\hlopt{:}\hlkwd{length}\hlstd{(W1)) \{}
  \hlstd{temp} \hlkwb{<-} \hlstd{W1[i]} \hlopt{+} \hlstd{W2[i]} \hlopt{+} \hlstd{W3[i]}
  \hlstd{b} \hlkwb{<-} \hlstd{b} \hlopt{+} \hlstd{temp} \hlopt{*} \hlstd{(s} \hlopt{-} \hlstd{temp)}
\hlstd{\}}

\hlstd{T} \hlkwb{<-} \hlstd{s} \hlopt{*} \hlstd{(s} \hlopt{-} \hlnum{1}\hlstd{)} \hlopt{*} \hlstd{a} \hlopt{/} \hlstd{b}
\hlstd{pval} \hlkwb{<-} \hlnum{1} \hlopt{-} \hlkwd{pchisq}\hlstd{(T, s} \hlopt{-} \hlnum{1}\hlstd{)}
\end{alltt}
\end{kframe}
\end{knitrout}
Значение критерия и p-value:
\begin{knitrout}
\definecolor{shadecolor}{rgb}{0.969, 0.969, 0.969}\color{fgcolor}\begin{kframe}
\begin{verbatim}
## [1] 3.0000000 0.2231302
\end{verbatim}
\end{kframe}
\end{knitrout}
p-value > 0.05, нет оснований отклонить гипотезу о неизменности уровня преступности



\end{document}
